% Vorlage für eine Bachelorarbeit
% Siehe auch LaTeX-Kurs von Mathematik-Online
% www.mathematik-online.org/kurse
% Anpassungen für die Fakultät für Mathematik
% am KIT durch Klaus Spitzmüller und Roland Schnaubelt Dezember 2011

\documentclass[12pt,a4paper]{scrartcl}
% scrartcl ist eine abgeleitete Artikel-Klasse im Koma-Skript
% zur Kontrolle des Umbruchs Klassenoption draft verwenden


% die folgenden Packete erlauben den Gebrauch von Umlauten und ß
% in der Latex Datei
\usepackage[utf8]{inputenc}
% \usepackage[latin1]{inputenc} %  Alternativ unter Windows
\usepackage[T1]{fontenc}
\usepackage[english]{babel}


\usepackage[pdftex]{graphicx}
\usepackage{latexsym}
\usepackage{amsmath,amssymb,amsthm}


% Abstand obere Blattkante zur Kopfzeile ist 2.54cm - 15mm
\setlength{\topmargin}{-15mm}


% Umgebungen für Definitionen, Sätze, usw.
% Es werden Sätze, Definitionen etc innerhalb einer Section mit
% 1.1, 1.2 etc durchnummeriert, ebenso die Gleichungen mit (1.1), (1.2) ..
\newtheorem{Satz}{Satz}[section]
\newtheorem{Definition}[Satz]{Definition} 
\newtheorem{Lemma}[Satz]{Lemma}	
\newtheorem{thm}{Theorem}
\newtheorem{prp}{Proposition}
                  
\numberwithin{equation}{section} 

% einige Abkuerzungen
\newcommand{\C}{\mathbb{C}} % komplexe
\newcommand{\K}{\mathbb{K}} % komplexe
\newcommand{\R}{\mathbb{R}} % reelle
\newcommand{\Q}{\mathbb{Q}} % rationale
\newcommand{\Z}{\mathbb{Z}} % ganze
\newcommand{\N}{\mathbb{N}} % natuerliche



\begin{document}
  % Keine Seitenzahlen im Vorspann
  \pagestyle{empty}

  % Titelblatt der Arbeit
  \begin{titlepage}

    \includegraphics[scale=0.45]{kit-logo.jpg} 
    \vspace*{2cm} 

 \begin{center} \large 
    
    Bachelors Thesis
    \vspace*{2cm}

    {\huge Markov Chain Monte Carlo for tall datasets}
    \vspace*{2.5cm}

    Fynn Orth
    \vspace*{1.5cm}

    Datum der Abgabe
    \vspace*{4.5cm}


    Betreuung: Mathias Trabs \\[1cm]
    Fakultät für Mathematik \\[1cm]
		Karlsruher Institut für Technologie
  \end{center}
\end{titlepage}



  % Inhaltsverzeichnis
  \tableofcontents

\newpage
 


  % Ab sofort Seitenzahlen in der Kopfzeile anzeigen
  \pagestyle{headings}

%%%%%%%%%%%%%%%%%%%%%%%%%%%%%%%%%
 \newpage  % neuer Abschnitt auf neue Seite, kann auch entfallen
%%%%%%%%%%%%%%%%%%%%%%%%%%%%%%%%%

\section{Bayesian Inference}
This section introduces the concept of Bayesian Inference which will be used to learn the parameters $\theta$ of a statistical model. \\
Bayesian inference is a subsection of statistical inference in which Bayes Theorem
is used to derive properties of a underlying distribution. 
 



\section{Markov Chains}
\begin{prp}
Consider a sequence $(a_n)$ of positive numbers which is converging to $a^*$ and a convergent series with running term $b_n$. The convolution 
\begin{align*}
    \lim_{n \rightarrow \infty} \sum_{j=1}^{n}a_j b_{n-j} = a^* \sum_{j=1}^{\infty}b_j.
\end{align*}
    
\end{prp}

\begin{thm}
[Convergence of Markov Chains]
Let $(X_n)$ be a Harris positive chain on $\mathcal{X}$ and denumerable, and  if there exists an ergodic atom $\alpha \subseteq \mathcal{X}$, then, for every $x \in \mathcal{X}$,
\begin{equation*}
    \lim_{n \rightarrow \infty} \| K^n(x,\cdot) - \pi \|_{TV} = 0
\end{equation*}
\end{thm}
\begin{proof}
We start by deriving composition formula \textit{"first entrance and last exist"} which relates $K^n(x,y)$ to the last visit to $\alpha$.
Intuitively, 
\begin{align*}
K^n(x,y) = P_x(X_n = y, \tau_\alpha \geq n) + P_x(X_n = y, \tau_\alpha < n).
\end{align*}
Furthermore
\begin{align*}
    P_x(X_n = y, \tau_\alpha < n) = \sum_{j=1}^{n-1} P_x(X_n = y, \tau_\alpha \leq j) \\
    = \sum_{j=1}^{n-1} \left [\sum_{k=1}^j P_x(X_k \in \alpha, \tau \geq k)K^{j-k}(\alpha, \alpha) \right ]
    P_\alpha(X_{n-j} = y, \tau_\alpha \geq n - j)
\end{align*}
Where $k$ is the time of entrance into $\alpha$ and $j$ is the time of exit from $\alpha$. \\
Using experssion (6.17) 
\begin{align}
    \pi (y) = \pi (\alpha) \sum_{j=1}^{\infty}P_\alpha(X_j = y, \tau_\alpha \geq j) \label{eq:inv_msr}
\end{align}
Combinning these two leads to 
\begin{align*}
    \| K^n(x,y) - \pi \|_{TV} = \sum_{y \in \mathcal{X}}| K^n(x,y) - \pi (y)| \\
    =  \sum_{y \in \mathcal{X}} P_x(X_n = y, \tau_\alpha \geq n) 
\\ + \sum_{y \in \mathcal{X}} \sum_{j=1}^{n-1} \sum_{k=1}^{j} |P_x(X_k \in \alpha, \tau = k) K^{j-k}(\alpha, \alpha) - \pi (\alpha)| 
\\ P_\alpha(X_{n-j} = y, \tau_\alpha \geq n-j) 
\\ + \sum_{y \in \mathcal{X}} \pi (\alpha) \sum_{j=n}^{\infty} P_\alpha(X_j = y, \tau_\alpha \geq j).
\end{align*}
From Harris Positivity and Proposition 6.33 we know that the first term goes to $0$ with n. The third term goes to zero as the tail of the convergent series from \eqref{eq:inv_msr}. Furthermore, the middle term can be rewritten as 
\begin{align}
    \sum_{y \in \mathcal{X}} \sum_{j=1}^{n-1} a_n b_{n-j} \label{eq:conv_prop}
\end{align}
 With the sequence $a_n = \sum_{k=1}^{n} |P_x(X_k \in \alpha, \tau = k) K^{n-k}(\alpha, \alpha) - \pi (\alpha)|$ converging to $0$ because $\alpha$ is an ergodic atom and $b_{n} = P_\alpha(X_{n} = y, \tau_\alpha \geq n)$ being a convergent series as mentioned in \eqref{eq:inv_msr}. The convergence of follows from Proposition 1
\end{proof}

  % Literaturverzeichnis (beginnt auf einer ungeraden Seite)
  \newpage

\begin{thebibliography}{Lam00}
 
\end{thebibliography}
 
      
  % ggf. hier Tabelle mit Symbolen 
  % (kann auch auf das Inhaltsverzeichnis folgen)

\newpage
  
 \thispagestyle{empty}


\vspace*{8cm}


\section*{Erkl\"arung}

Ich  versichere  wahrheitsgem\"a\ss,  die  Arbeit selbstst\"andig verfasst,  alle  benutzten  Hilfsmittel  vollst\"andig  und  genau  angegeben  und  alles kenntlich  gemacht  zu  haben,  was  aus  Arbeiten  anderer  unver\"andert  oder  mit  Ab\"anderungen entnommen  wurde,  sowie die Satzung  des  KIT  zur  Sicherung guter wissenschaftlicher Praxis in der jeweils g\"ultigen Fassung beachtet zu haben.
\\[2ex] 

\noindent
Ort, den Datum\\[5ex]

% Unterschrift (handgeschrieben)



\end{document}

